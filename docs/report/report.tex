
% See http://tex.stackexchange.com/questions/102662/harvard-reference-using-biblatex

\documentclass[a4paper]{article}

\usepackage[english]{babel}
\usepackage[utf8]{inputenc}
\usepackage{amsmath}
\usepackage{csquotes} % Recommended

\usepackage[style=authoryear-ibid,backend=biber]{biblatex}

%\addbibresource{dementiaVA.bib}

\title{Developing a Voice Assistant for People Living with Dementia}
\author{Kieron Gillingham}
\date{May 2021}

\begin{document}
\maketitle

\begin{abstract}
Dementia is one of the leading causes of disability among the global elderly population which causes the deterioration of cognitive functioning. People living with dementia require constant care; however the number of caregivers is not sufficient to match the growing number of dementia cases. There is great promise for voice assistants (VAs) such as Apple’s Siri and Amazon’s Alexa to support care-givers and patients by managing routine tasks such as setting medication reminders, carrying out mental stimulation exercises, and alerting human carers when needed. This project aims to develop a prototype VA that is tailored for users living with dementia.
\end{abstract}

\section*{Introduction}
Dementia is one of the leading causes of disability among the global elderly population which causes the deterioration of cognitive functioning. People living with dementia require constant care; however the number of caregivers is not sufficient to match the growing number of dementia cases. There is great promise for voice assistants (VAs) such as Apple’s Siri and Amazon’s Alexa to support care-givers and patients by managing routine tasks such as setting medication reminders, carrying out mental stimulation exercises, and alerting human carers when needed. This project aims to develop a prototype VA that is tailored for users living with dementia.

%\printbibliography

\end{document}
